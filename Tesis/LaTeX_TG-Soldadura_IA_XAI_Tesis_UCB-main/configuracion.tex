%---------------------------------------------------------------------
% Archivo de configuración: configuracion.tex
%---------------------------------------------------------------------

%=====================================================================
% Paquetes básicos y configuración de idioma/tipografía
%=====================================================================
\usepackage[left=3.5cm, right=3cm, top=3cm, bottom=3cm]{geometry} % Márgenes
\usepackage[english,main=spanish]{babel} % Idiomas 
\addto\captionsspanish{\renewcommand{\tablename}{Tabla}}
\captionsetup[table]{%
	labelsep=newline,            % Salto de línea entre "Tabla 1" y el título
	labelfont=bf,                % "Tabla 1" en negrita
	textfont=it,                 % Título en cursiva
	justification=raggedright,   % Alineado a la izquierda
	singlelinecheck=false        % Evitar centrado si es una sola línea
}

\usepackage[T1]{fontenc}       % Codificación de fuente
\usepackage{times}             % Usa Times New Roman
\usepackage{fontsize}          % Permite comandos de tamaño de fuente
\usepackage{xcolor}            % Colores
\usepackage{graphicx}          % Manejo de gráficos
\graphicspath{{./img/}}        % Ruta por defecto para las imágenes

%=====================================================================
% Paquetes adicionales
%=====================================================================
\usepackage{tocloft}           % Personalizar tabla de contenidos, figuras, tablas
\usepackage{fancyhdr}          % Encabezados y pies de página
\pagestyle{fancy}
\fancyhf{}
\rfoot{\thepage}               % Número de página alineado a la derecha
\renewcommand{\headrulewidth}{0pt} % Eliminar línea de encabezado

%=====================================================================
% Configuración de interlineado
%=====================================================================
\renewcommand{\baselinestretch}{1.5} % Interlineado 1.5

%=====================================================================
% Definición de macros para datos del documento
%=====================================================================
\newcommand{\titulo}[1]{\def\Titulo{#1}}
\newcommand{\autor}[1]{\def\Autor{#1}}
\newcommand{\tutor}[1]{\def\Tutor{#1}}
\newcommand{\fecha}[1]{\def\Fecha{#1}}
\newcommand{\departamento}[1]{\def\Departamento{#1}}
\newcommand{\carrera}[1]{\def\Carrera{#1}}
\newcommand{\fechaActual}{\number\year}  % Muestra únicamente el año actual

%=====================================================================
% Comandos para carátulas (portadas)
%=====================================================================

% Carátula simple (sin tutor)
\newcommand{\caratulaTapa}{
	\begin{titlepage}
		\begin{center}
			{\fontsize{18}{20}\selectfont UNIVERSIDAD CATÓLICA BOLIVIANA "SAN PABLO" SEDE TARIJA}\\[0.5cm]
			{\fontsize{16}{18}\selectfont DEPARTAMENTO DE \MakeUppercase{\Departamento}}\\
			{\fontsize{14}{16}\selectfont CARRERA DE: \MakeUppercase{\Carrera}\\}
			\begin{figure}[h]
				\centering
				\includegraphics[height=7cm]{ucbLOGO}
			\end{figure}
			{\fontsize{16}{18}\selectfont \MakeUppercase{\Titulo} }
			\vspace{1cm}
			
			{\fontsize{14}{16}\selectfont POSTULANTE: \MakeUppercase{\Autor}\\}
		\end{center}
		\vspace{0.2cm}
		{\fontsize{12}{14}\selectfont
			Trabajo de Tesis de grado presentado en consideración de la Universidad Católica Boliviana "San Pablo", como requisito para optar el Grado Académico de Licenciatura en \Carrera 
			
		}
		
		{\centering \fontsize{14}{16}\selectfont TARIJA-BOLIVIA\\\Fecha\\}
		
		\thispagestyle{empty} % Evita numeración en la portada
	\end{titlepage}
}

% Carátula que incluye tutor
\newcommand{\caratulaContenido}{
	\begin{titlepage}
		\begin{center}
			{\fontsize{18}{20}\selectfont UNIVERSIDAD CATÓLICA BOLIVIANA "SAN PABLO" SEDE TARIJA}\\[0.5cm]
			{\fontsize{16}{18}\selectfont DEPARTAMENTO DE \MakeUppercase{\Departamento}}\\
			{\fontsize{14}{16}\selectfont CARRERA DE: \MakeUppercase{\Carrera}\\}
			\begin{figure}[h]
				\centering
				\includegraphics[height=7cm]{ucbLOGO}
			\end{figure}
			{\fontsize{16}{18}\selectfont \MakeUppercase{\Titulo}}
			\vspace{1cm}
			
			{\fontsize{14}{16}\selectfont POSTULANTE: \MakeUppercase{\Autor}\\
				TUTOR: \MakeUppercase{\Tutor}}
		\end{center}
		\vspace{0.2cm}
		{\fontsize{12}{14}\selectfont
			Trabajo de Tesis de grado presentado en consideración de la Universidad Católica Boliviana "San Pablo", como requisito para optar el Grado Académico de Licenciatura en \Carrera 
		}
		
		{\centering \fontsize{14}{16}\selectfont TARIJA-BOLIVIA\\\Fecha\\}
		
		\thispagestyle{empty} % Evita numeración en la portada
	\end{titlepage}
}

%=====================================================================
% Comando para iniciar numeración a partir de la introducción
%=====================================================================
\newcommand{\iniciarNumeracion}{
	\setcounter{page}{1}   % Reinicia la numeración de páginas
	\pagestyle{fancy}      % Usa el estilo "fancy"
	\fancyhf{}             % Limpiar encabezados y pies
	\fancyfoot[R]{\thepage}          % Número de página a la derecha
	\renewcommand{\headrulewidth}{0pt} % Sin línea de encabezado
	\renewcommand{\footrulewidth}{0pt} % Sin línea de pie de página
	
	% Redefinir el estilo 'plain' para páginas de inicio de capítulo
	\fancypagestyle{plain}{%
		\fancyhf{}
		\fancyfoot[R]{\thepage}
		\renewcommand{\headrulewidth}{0pt}
		\renewcommand{\footrulewidth}{0pt}
	}
}

%=====================================================================
% Comando para configurar índices (ToC, LoF, LoT)
%=====================================================================
\newcommand{\configurarIndices}{
	\setcounter{tocdepth}{3}      % Profundidad del índice de contenidos
	\setcounter{secnumdepth}{3}   % Profundidad de numeración de secciones
	
	% Puntos (dotfill) en los niveles de índice
	\renewcommand{\cftchapleader}{\cftdotfill{\cftdotsep}}
	\renewcommand{\cftsecleader}{\cftdotfill{\cftdotsep}}
	\renewcommand{\cftsubsecleader}{\cftdotfill{\cftdotsep}}
	\renewcommand{\cftsubsubsecleader}{\cftdotfill{\cftdotsep}}
	
	% Opcional: líder de puntos para figuras y tablas
	\renewcommand{\cftfigleader}{\cftdotfill{\cftdotsep}}
	\renewcommand{\cfttableader}{\cftdotfill{\cftdotsep}}
}

% Fin de configuracion.tex
